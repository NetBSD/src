%%%%%%%%%%%%%%%% gdb-refcard.tex %%%%%%%%%%%%%%%%

%This file is TeX source for a reference card describing GDB, the GNU debugger.
%Copyright (C) 1991-2014 Free Software Foundation, Inc.
%Permission is granted to make and distribute verbatim copies of
%this reference provided the copyright notices and permission notices
%are preserved on all copies.
%
%TeX markup is a programming language; accordingly this file is source
%for a program to generate a reference.
%
%This program is free software; you can redistribute it and/or modify
%it under the terms of the GNU General Public License as published by
%the Free Software Foundation; either version 3, or (at your option)
%any later version.
%
%This program is distributed in the hope that it will be useful, but
%WITHOUT ANY WARRANTY; without even the implied warranty of
%MERCHANTABILITY or FITNESS FOR A PARTICULAR PURPOSE.  See the GNU
%General Public License for more details.
%
%You should have received a copy of the GNU General Public License
%along with this program.  If not, see <http://www.gnu.org/licenses/>.
%
%You can contact the maintainer at:  doc@cygnus.com
%
%                                Documentation Department
%                                Cygnus Solutions
%                                1325 Chesapeake Terrace
%                                Sunnyvale, CA 94089  USA
%
%                                +1 800 CYGNUS-1
%
%
%
% 22-AUG-1993 Andreas Vogel
%
%   Modifications made in order to handle different papersizes correctly.
%   You only have to set the total width and height of the paper, the
%   horizontal and vertical margin space measured from *paper edge*
%   and the interline and interspec spacing.
%   In order to support a new papersize, you have to fiddle with the
%   latter four dimensions. Just try out a few values.
%   All other values will be computed at process time so it should be
%   quite easy to support different paper sizes - only four values to
%   guess :-)
%
%   To find the configuration places, just search for the string
%   "CONFIGURATION".
%
%   Andreas Vogel (av@ssw.de)
%
%
%
% Uncomment the following `magnification' command if you want to print
% out in a larger font.  Caution!  You may need larger paper.  You had
% best avoid using 3-column output if you try this.  See the ``Three
% column format'' section below if you want to print in three column
% format.
%
%\magnification=\magstep 1
%
% NOTE ON INTENTIONAL OMISSIONS: This reference card includes most GDB
% commands, but due to space constraints there are some things I chose
% to omit.  In general, not all synonyms for commands are covered, nor
% all variations of a command.
% The GDB-under-Emacs section omits gdb-mode functions without default
% keybindings.  GDB startup options are not described.
% set print sevenbit-strings omitted.
% printsyms, printpsyms, omitted since they're for GDB maintenance primarily
% share omitted due to obsolescence
% set check range/type omitted at least til code is in GDB.
%
%-------------------- Three column format -----------------------

%%%% --- To disable three column format, comment out this entire section

% Three-column format for landscape printing

%-------- Papersize defs:

\newdimen\totalwidth \newdimen\totalheight
\newdimen\hmargin    \newdimen\vmargin
\newdimen\secskip    \newdimen\lskip
\newdimen\barwidth   \newdimen\barheight
\newdimen\intersecwidth

%%
%%  START CONFIGURATION - PAPERSIZE DEFINITIONS
%------- Papersize params:
%%  US letter paper (8.5x11in)
%%
\totalwidth=11in    % total width of paper
\totalheight=8.5in  % total height of paper
\hmargin=.25in      % horizontal margin width
\vmargin=.25in      % vertical margin width
\secskip=1pc        % space between refcard secs
\lskip=2pt          % extra skip between \sec entries
\ifx\pdfoutput\undefined\else   % check if we are using pdfTeX
  \pdfpagewidth=\totalwidth     % width of paper in pdf output
  \pdfpageheight=\totalheight   % height of paper in pdf output
\fi
%------- end papersize params
%%
%%  change according to personal taste, not papersize dependent
%%
\barwidth=.1pt       % width of the cropmark bar
\barheight=2pt       % height of the cropmark bar
\intersecwidth=0.5em % width between \itmwid and \dfnwid
%%
%%  END CONFIGURATION - PAPERSIZE DEFINITIONS
%%

%%
%%  values to be computed - nothing to configure
%%
\newdimen\fullhsize     % width of area without margins
\newdimen\itmwid        % width of item column
\newdimen\dfnwid        % width of definition column
\newdimen\temp          % only for temporary use

%%
%%  adjust the offsets so the margins are measured *from paper edge*
%%
\hoffset=-1in \advance \hoffset by \hmargin
\voffset=-1in \advance \voffset by \vmargin

%%
%%  fullhsize = totalwidth - (2 * hmargin)
%%
\fullhsize=\totalwidth
\temp=\hmargin \multiply \temp by 2 \advance \fullhsize by -\temp

%%
%%  hsize = (fullhsize - (4 * hmargin) - (2 * barwidth)) / 3
%%
\hsize=\fullhsize
\temp=\hmargin \multiply \temp by 4 \advance \hsize by -\temp
\temp=\barwidth \multiply \temp by 2 \advance \hsize by -\temp
\divide \hsize by 3

%%
%%  vsize = totalheight - (2 * vmargin)
%%
\vsize=\totalheight
\temp=\vmargin \multiply \temp by 2 \advance \vsize by -\temp

%%
%%  itmwid = (hsize - intersecwidth) * 1/3
%%  dfnwid = (hsize - intersecwidth) * 2/3
%%
\temp=\hsize \advance \temp by -\intersecwidth \divide \temp by 3
\itmwid=\temp
\dfnwid=\hsize \advance \dfnwid by -\itmwid

%-------- end papersize defs


\def\fulline{\hbox to \fullhsize}
\let\lcr=L \newbox\leftcolumn\newbox\centercolumn
\output={\if L\lcr
            \global\setbox\leftcolumn=\columnbox \global\let\lcr=C
	 \else
            \if C\lcr
               \global\setbox\centercolumn=\columnbox \global\let\lcr=R
            \else \tripleformat \global\let\lcr=L
            \fi
         \fi
%         \ifnum\outputpenalty>-20000 \else\dosupereject\fi
      }

%%
%%  START CONFIGURATION - ALTERNATIVE FOLDING GUIDES
%%
%%  For NO printed folding guide,
%%  comment out other \def\vdecor's and uncomment:

%\def\vdecor{\hskip\hmargin plus1fil\hskip\barwidth plus1fil\hskip\hmargin plus1fil}

%%  For SOLID LINE folding guide,
%%  comment out other \def\vdecor's and uncomment:

%\def\vdecor{\hskip\hmargin plus1fil \vrule width \barwidth \hskip\hmargin plus1fil}

%%  For SMALL MARKS NEAR TOP AND BOTTOM as folding guide,
%%  comment out other \def\vdecor's and uncomment:

\def\vdecor{\hskip\hmargin plus1fil
\vbox to \vsize{\hbox to \barwidth{\vrule height\barheight width\barwidth}\vfill
\hbox to \barwidth{\vrule height\barheight width\barwidth}}%THIS PERCENT SIGN IS ESSENTIAL
\hskip\hmargin plus1fil}

%%
%%  END CONFIGURATION - ALTERNATIVES FOR FOLDING GUIDES
%%

\def\tripleformat{\shipout\vbox{\fulline{\box\leftcolumn\vdecor
					 \box\centercolumn\vdecor
					 \columnbox}
			       }
                 \advancepageno}
\def\columnbox{\leftline{\pagebody}}
\def\bye{\par\vfill
         \supereject
         \if R\lcr \null\vfill\eject\fi
         \end}

%-------------------- end three column format -----------------------

%-------------------- Computer Modern font defs: --------------------
\font\bbf=cmbx10
\font\vbbf=cmbx12
\font\smrm=cmr6
\font\brm=cmr10
\font\rm=cmr7
\font\it=cmti7
\font\tt=cmtt8
%-------------------- end font defs ---------------------------------

%
\hyphenpenalty=5000\tolerance=2000\raggedright\raggedbottom
\normalbaselineskip=9pt\baselineskip=9pt
%
\parindent=0pt
\parskip=0pt
\footline={\vbox to0pt{\hss}}
%
\def\ctl#1{{\tt C-#1}}
\def\opt#1{{\brm[{\rm #1}]}}
\def\xtra#1{\noalign{\smallskip{\tt#1}}}
%
\long\def\sec#1;#2\endsec{\vskip \secskip
\halign{%
%COL 1 (of halign):
\vtop{\hsize=\itmwid\tt
##\par\vskip \lskip }\hfil
%COL 2 (of halign):
&\vtop{\hsize=\dfnwid\hangafter=1\hangindent=\intersecwidth
\rm ##\par\vskip \lskip}\cr
%Tail of \long\def fills in halign body with \sec args:
\noalign{{\bbf #1}\vskip \lskip}
#2
}
}

{\vbbf GDB QUICK REFERENCE}\hfil{\smrm GDB Version 5}\qquad

\sec Essential Commands;
gdb {\it program} \opt{{\it core}}&debug {\it program} \opt{using
coredump {\it core}}\cr
b \opt{\it file\tt:}{\it function}&set breakpoint at {\it function} \opt{in \it file}\cr
run \opt{{\it arglist}}&start your program \opt{with {\it arglist}}\cr
bt& backtrace: display program stack\cr
p {\it expr}&display the value of an expression\cr
c &continue running your program\cr
n &next line, stepping over function calls\cr
s &next line, stepping into function calls\cr
\endsec

\sec Starting GDB;
gdb&start GDB, with no debugging files\cr
gdb {\it program}&begin debugging {\it program}\cr
gdb {\it program core}&debug coredump {\it core} produced by {\it
program}\cr
gdb --help&describe command line options\cr
\endsec

\sec Stopping GDB;
quit&exit GDB; also {\tt q} or {\tt EOF} (eg \ctl{d})\cr
INTERRUPT&(eg \ctl{c}) terminate current command, or send to running process\cr
\endsec

\sec Getting Help;
help&list classes of commands\cr
help {\it class}&one-line descriptions for commands in {\it class}\cr
help {\it command}&describe {\it command}\cr
\endsec

\sec Executing your Program;
run {\it arglist}&start your program with {\it arglist}\cr
run&start your program with current argument list\cr
run $\ldots$ <{\it inf} >{\it outf}&start your program with input, output
redirected\cr
\cr
kill&kill running program\cr
\cr
tty {\it dev}&use {\it dev} as stdin and stdout for next {\tt run}\cr
set args {\it arglist}&specify {\it arglist} for next
{\tt run}\cr
set args&specify empty argument list\cr
show args&display argument list\cr
\cr
show env&show all environment variables\cr
show env {\it var}&show value of environment variable {\it var}\cr
set env {\it var} {\it string}&set environment variable {\it var}\cr
unset env {\it var}&remove {\it var} from environment\cr
\endsec

\sec Shell Commands;
cd {\it dir}&change working directory to {\it dir}\cr
pwd&Print working directory\cr
make $\ldots$&call ``{\tt make}''\cr
shell {\it cmd}&execute arbitrary shell command string\cr
\endsec

\vfill
\line{\smrm \opt{ } surround optional arguments \hfill $\ldots$ show
one or more arguments}
\vskip\baselineskip
\centerline{\smrm \copyright 1998-2014 Free Software Foundation, Inc.\qquad Permissions on back}
\eject
\sec Breakpoints and Watchpoints;
break \opt{\it file\tt:}{\it line}\par
b \opt{\it file\tt:}{\it line}&set breakpoint at {\it line} number \opt{in \it file}\par 
eg:\quad{\tt break main.c:37}\quad\cr
break \opt{\it file\tt:}{\it func}&set breakpoint at {\it
func} \opt{in \it file}\cr
break +{\it offset}\par
break -{\it offset}&set break at {\it offset} lines from current stop\cr
break *{\it addr}&set breakpoint at address {\it addr}\cr
break&set breakpoint at next instruction\cr
break $\ldots$ if {\it expr}&break conditionally on nonzero {\it expr}\cr
cond {\it n} \opt{\it expr}&new conditional expression on breakpoint
{\it n}; make unconditional if no {\it expr}\cr
tbreak $\ldots$&temporary break; disable when reached\cr
rbreak \opt{\it file\tt:}{\it regex}&break on all functions matching {\it
regex} \opt{in \it file}\cr
watch {\it expr}&set a watchpoint for expression {\it expr}\cr
catch {\it event}&break at {\it event}, which may be {\tt catch}, {\tt throw},
{\tt exec}, {\tt fork}, {\tt vfork}, {\tt load}, or {\tt unload}.\cr
\cr
info break&show defined breakpoints\cr
info watch&show defined watchpoints\cr
\cr
clear&delete breakpoints at next instruction\cr
clear \opt{\it file\tt:}{\it fun}&delete breakpoints at entry to {\it fun}()\cr
clear \opt{\it file\tt:}{\it line}&delete breakpoints on source line \cr
delete \opt{{\it n}}&delete breakpoints
\opt{or breakpoint {\it n}}\cr
\cr
disable \opt{{\it n}}&disable breakpoints
\opt{or breakpoint {\it n}}
\cr
enable \opt{{\it n}}&enable breakpoints 
\opt{or breakpoint {\it n}}
\cr
enable once \opt{{\it n}}&enable breakpoints \opt{or breakpoint {\it n}}; 
disable again when reached
\cr
enable del \opt{{\it n}}&enable breakpoints \opt{or breakpoint {\it n}}; 
delete when reached
\cr
\cr
ignore {\it n} {\it count}&ignore breakpoint {\it n}, {\it count}
times\cr
\cr
commands {\it n}\par
\qquad \opt{\tt silent}\par
\qquad {\it command-list}&execute GDB {\it command-list} every time breakpoint {\it n} is reached. \opt{{\tt silent} suppresses default
display}\cr
end&end of {\it command-list}\cr
\endsec

\sec Program Stack;
backtrace \opt{\it n}\par
bt \opt{\it n}&print trace of all frames in stack; or of {\it n}
frames---innermost if {\it n}{\tt >0}, outermost if {\it n}{\tt <0}\cr
frame \opt{\it n}&select frame number {\it n} or frame at address {\it
n}; if no {\it n}, display current frame\cr
up {\it n}&select frame {\it n} frames up\cr
down {\it n}&select frame {\it n} frames down\cr
info frame \opt{\it addr}&describe selected frame, or frame at
{\it addr}\cr
info args&arguments of selected frame\cr
info locals&local variables of selected frame\cr
info reg \opt{\it rn}$\ldots$\par
info all-reg \opt{\it rn}&register values \opt{for regs {\it rn\/}} in
selected frame; {\tt all-reg} includes floating point\cr
\endsec

\vfill\eject
\sec Execution Control;
continue \opt{\it count}\par
c \opt{\it count}&continue running; if {\it count} specified, ignore
this breakpoint next {\it count} times\cr
\cr
step \opt{\it count}\par
s \opt{\it count}&execute until another line reached; repeat {\it count} times if
specified\cr
stepi \opt{\it count}\par
si \opt{\it count}&step by machine instructions rather than source
lines\cr
\cr
next \opt{\it count}\par
n \opt{\it count}&execute next line, including any function calls\cr
nexti \opt{\it count}\par
ni \opt{\it count}&next machine instruction rather than source
line\cr
\cr
until \opt{\it location}&run until next instruction (or {\it
location})\cr
finish&run until selected stack frame returns\cr
return \opt{\it expr}&pop selected stack frame without executing
\opt{setting return value}\cr
signal {\it num}&resume execution with signal {\it s} (none if {\tt 0})\cr
jump {\it line}\par
jump *{\it address}&resume execution at specified {\it line} number or
{\it address}\cr
set var={\it expr}&evaluate {\it expr} without displaying it; use for
altering program variables\cr
\endsec

\sec Display;
print \opt{\tt/{\it f}\/} \opt{\it expr}\par
p \opt{\tt/{\it f}\/} \opt{\it expr}&show value of {\it expr} \opt{or
last value \tt \$} according to format {\it f}:\cr
\qquad x&hexadecimal\cr
\qquad d&signed decimal\cr
\qquad u&unsigned decimal\cr
\qquad o&octal\cr
\qquad t&binary\cr
\qquad a&address, absolute and relative\cr
\qquad c&character\cr
\qquad f&floating point\cr
call \opt{\tt /{\it f}\/} {\it expr}&like {\tt print} but does not display
{\tt void}\cr
x \opt{\tt/{\it Nuf}\/} {\it expr}&examine memory at address {\it expr};
optional format spec follows slash\cr
\quad {\it N}&count of how many units to display\cr
\quad {\it u}&unit size; one of\cr
&{\tt\qquad b}\ individual bytes\cr
&{\tt\qquad h}\ halfwords (two bytes)\cr
&{\tt\qquad w}\ words (four bytes)\cr
&{\tt\qquad g}\ giant words (eight bytes)\cr
\quad {\it f}&printing format.  Any {\tt print} format, or\cr
&{\tt\qquad s}\ null-terminated string\cr
&{\tt\qquad i}\ machine instructions\cr
disassem \opt{\it addr}&display memory as machine instructions\cr
\endsec

\sec Automatic Display;
display \opt{\tt/\it f\/} {\it expr}&show value of {\it expr} each time
program stops \opt{according to format {\it f}\/}\cr
display&display all enabled expressions on list\cr
undisplay {\it n}&remove number(s) {\it n} from list of
automatically displayed expressions\cr
disable disp {\it n}&disable display for expression(s) number {\it
n}\cr
enable disp {\it n}&enable display for expression(s) number {\it
n}\cr
info display&numbered list of display expressions\cr
\endsec

\vfill\eject

\sec Expressions;
{\it expr}&an expression in C, C++, or Modula-2 (including function calls), or:\cr
{\it addr\/}@{\it len}&an array of {\it len} elements beginning at {\it
addr}\cr
{\it file}::{\it nm}&a variable or function {\it nm} defined in {\it
file}\cr
$\tt\{${\it type}$\tt\}${\it addr}&read memory at {\it addr} as specified
{\it type}\cr
\$&most recent displayed value\cr
\${\it n}&{\it n}th displayed value\cr
\$\$&displayed value previous to \$\cr
\$\${\it n}&{\it n}th displayed value back from \$\cr
\$\_&last address examined with {\tt x}\cr
\$\_\_&value at address \$\_\cr
\${\it var}&convenience variable; assign any value\cr
\cr
show values \opt{{\it n}}&show last 10 values \opt{or surrounding
\${\it n}}\cr
show conv&display all convenience variables\cr
\endsec

\sec Symbol Table;
info address {\it s}&show where symbol {\it s} is stored\cr
info func \opt{\it regex}&show names, types of defined functions
(all, or matching {\it regex})\cr
info var \opt{\it regex}&show names, types of global variables (all,
or matching {\it regex})\cr
whatis \opt{\it expr}\par
ptype \opt{\it expr}&show data type of {\it expr} \opt{or \tt \$}
without evaluating; {\tt ptype} gives more detail\cr
ptype {\it type}&describe type, struct, union, or enum\cr
\endsec

\sec GDB Scripts;
source {\it script}&read, execute GDB commands from file {\it
script}\cr
\cr
define {\it cmd}\par
\qquad {\it command-list}&create new GDB command {\it cmd}; 
execute script defined by {\it command-list}\cr
end&end of {\it command-list}\cr
document {\it cmd}\par
\qquad {\it help-text}&create online documentation 
for new GDB command {\it cmd}\cr
end&end of {\it help-text}\cr
\endsec

\sec Signals;
handle {\it signal} {\it act}&specify GDB actions for {\it signal}:\cr
\quad print&announce signal\cr
\quad noprint&be silent for signal\cr
\quad stop&halt execution on signal\cr
\quad nostop&do not halt execution\cr
\quad pass&allow your program to handle signal\cr
\quad nopass&do not allow your program to see signal\cr
info signals&show table of signals, GDB action for each\cr
\endsec

\sec Debugging Targets;
target {\it type} {\it param}&connect to target machine, process, or file\cr
help target&display available targets\cr
attach {\it param}&connect to another process\cr
detach&release target from GDB control\cr
\endsec

\vfill\eject
\sec Controlling GDB;
set {\it param} {\it value}&set one of GDB's internal parameters\cr
show {\it param}&display current setting of parameter\cr
\xtra{\rm Parameters understood by {\tt set} and {\tt show}:}
\quad complaint {\it limit}&number of messages on unusual symbols\cr
\quad confirm {\it on/off}&enable or disable cautionary queries\cr
\quad editing {\it on/off}&control {\tt readline} command-line editing\cr
\quad height {\it lpp}&number of lines before pause in display\cr
\quad language {\it lang}&Language for GDB expressions ({\tt auto}, {\tt c} or
{\tt modula-2})\cr
\quad listsize {\it n}&number of lines shown by {\tt list}\cr
\quad prompt {\it str}&use {\it str} as GDB prompt\cr
\quad radix {\it base}&octal, decimal, or hex number representation\cr
\quad verbose {\it on/off}&control messages when loading
symbols\cr
\quad width {\it cpl}&number of characters before line folded\cr
\quad write {\it on/off}&Allow or forbid patching binary, core files
(when reopened with {\tt exec} or {\tt core})
\cr
\quad history $\ldots$\par
\quad h $\ldots$&groups with the following options:\cr
\quad h exp {\it off/on}&disable/enable {\tt readline} history expansion\cr
\quad h file {\it filename}&file for recording GDB command history\cr
\quad h size {\it size}&number of commands kept in history list\cr
\quad h save {\it off/on}&control use of external file for
command history\cr
\cr
\quad print $\ldots$\par
\quad p $\ldots$&groups with the following options:\cr
\quad p address {\it on/off}&print memory addresses in stacks,
values\cr
\quad p  array {\it off/on}&compact or attractive format for
arrays\cr
\quad p demangl {\it on/off}&source (demangled) or internal form for C++
symbols\cr
\quad p asm-dem {\it on/off}&demangle C++ symbols in
machine-instruction output\cr
\quad p elements {\it limit}&number of array elements to display\cr
\quad p object {\it on/off}&print C++ derived types for objects\cr
\quad p pretty {\it off/on}&struct display: compact or indented\cr
\quad p union {\it on/off}&display of union members\cr
\quad p vtbl {\it off/on}&display of C++ virtual function
tables\cr
\cr
show commands&show last 10 commands\cr
show commands {\it n}&show 10 commands around number {\it n}\cr
show commands +&show next 10 commands\cr
\endsec

\sec Working Files;
file \opt{\it file}&use {\it file} for both symbols and executable;
with no arg, discard both\cr
core \opt{\it file}&read {\it file} as coredump; or discard\cr
exec \opt{\it file}&use {\it file} as executable only; or discard\cr
symbol \opt{\it file}&use symbol table from {\it file}; or discard\cr
load {\it file}&dynamically link {\it file\/} and add its symbols\cr
add-sym {\it file} {\it addr}&read additional symbols from {\it file},
dynamically loaded at {\it addr}\cr
info files&display working files and targets in use\cr
path {\it dirs}&add {\it dirs} to front of path searched for
executable and symbol files\cr
show path&display executable and symbol file path\cr
info share&list names of shared libraries currently loaded\cr
\endsec

\vfill\eject
\sec Source Files;
dir {\it names}&add directory {\it names} to front of source path\cr
dir&clear source path\cr
show dir&show current source path\cr
\cr
list&show next ten lines of source\cr
list -&show previous ten lines\cr
list {\it lines}&display source surrounding {\it lines}, 
specified as:\cr
\quad{\opt{\it file\tt:}\it num}&line number \opt{in named file}\cr
\quad{\opt{\it file\tt:}\it function}&beginning of function \opt{in
named file}\cr
\quad{\tt +\it off}&{\it off} lines after last printed\cr
\quad{\tt -\it off}&{\it off} lines previous to last printed\cr
\quad{\tt*\it address}&line containing {\it address}\cr
list {\it f},{\it l}&from line {\it f} to line {\it l}\cr
info line {\it num}&show starting, ending addresses of compiled code for
source line {\it num}\cr
info source&show name of current source file\cr
info sources&list all source files in use\cr
forw {\it regex}&search following source lines for {\it regex}\cr
rev {\it regex}&search preceding source lines for {\it regex}\cr
\endsec

\sec GDB under GNU Emacs;
M-x gdb&run GDB under Emacs\cr
\ctl{h} m&describe GDB mode\cr
M-s&step one line ({\tt step})\cr
M-n&next line ({\tt next})\cr
M-i&step one instruction ({\tt stepi})\cr
\ctl{c} \ctl{f}&finish current stack frame ({\tt finish})\cr
M-c&continue ({\tt cont})\cr
M-u&up {\it arg} frames ({\tt up})\cr
M-d&down {\it arg} frames ({\tt down})\cr
\ctl{x} \&&copy number from point, insert at end\cr
\ctl{x} SPC&(in source file) set break at point\cr
\endsec

\sec GDB License;
show copying&Display GNU General Public License\cr
show warranty&There is NO WARRANTY for GDB.  Display full no-warranty
statement.\cr 
\endsec


\vfill
{\smrm\parskip=6pt
Copyright \copyright 1991-2014 Free Software Foundation, Inc.
Author: Roland H. Pesch

The author assumes no responsibility for any errors on this card.

This card may be freely distributed under the terms of the GNU
General Public License.

Please contribute to development of this card by
annotating it.  Improvements can be sent to bug-gdb@gnu.org.

GDB itself is free software; you are welcome to distribute copies of
it under the terms of the GNU General Public License.  There is
absolutely no warranty for GDB.
}
\end
